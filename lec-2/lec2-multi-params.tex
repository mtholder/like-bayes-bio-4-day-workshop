\documentclass[landscape]{foils} 
\usepackage[pdftex]{graphicx}
\DeclareGraphicsExtensions{.pdf, .jpg, .tif, .png}
%\usepackage{pslatex}
\usepackage{tabularx,dcolumn, graphicx, amsfonts,amsmath}  
\usepackage[sectionbib]{natbib}
\usepackage{picinpar}
\usepackage{paralist}
\usepackage[sectionbib]{natbib}
\bibliographystyle{apalike}
\setlength{\voffset}{-0.5in}
%\setlength{\hoffset}{-0.5in}
%\setlength{\textwidth}{10.5in}
\setlength{\textheight}{7in}
\setlength{\parindent}{0pt}
%\pagestyle{empty}
%\renewcommand{\baselinestretch}{2.0}
\DeclareMathSymbol{\expect}{\mathalpha}{AMSb}{'105}
\def\p{\rm p}
\def\pp{\rm P}
\newcommand{\section}{\secdef \newsection\newsection}
%\renewcommand{\labelitemi}{\includegraphics[width=5mm]{images/bullet.pdf}}
\newcommand{\newsection}[1]{%
{
	\par\flushleft\large\sf\bfseries \vskip -2cm #1\\\rule[0.7\baselineskip]{\textwidth}{0.5mm}\par}}

\newcommand{\subsection}{\secdef \test\test}
\newcommand{\test}[1]{%
	{\par\flushleft\normalsize\sf\bfseries #1: }}
\newcommand{\M}{\mathcal{M}}
\newcommand{\prob}{{\rm Prob~}}
\def\showy#1{{\normalsize\sf\bfseries #1}}
\def\donotuse#1{}

\newcommand{\entrylabel}[1]{\mbox{#1}\hfil}
\newenvironment{entry}
	{\begin{list}{}%
		{\renewcommand{\makelabel}{\entrylabel}%
		\setlength{\labelwidth}{35pt}%
		\setlength{\leftmargin}{\labelwidth+\labelsep}%
	}%
	{\end{list}}}
% this are commands that come with the color package

\usepackage{color}
\usepackage{fancyhdr}


\pagestyle{empty}
%define colors
\definecolor{mediumblue}{rgb}{0.0509,0.35,0.568}
\definecolor{blue}{rgb}{0.0109,0.15,0.468}
\definecolor{black}{rgb}{0.04,0.06,0.2}
\definecolor{darkblue}{rgb}{0.03,0.1,0.2}
\definecolor{darkgreen}{rgb}{0.03,0.5,0.2}
\definecolor{lightblue}{rgb}{0.85,0.9333,0.95}
\definecolor{lightblue2}{rgb}{0.270588, 0.45098, 0.701961}
\definecolor{white}{rgb}{1.0,1.0,1.0}
\definecolor{yellow}{rgb}{0.961,0.972,0.047}
\definecolor{red}{rgb}{0.9,0.1,0.1}
\definecolor{orange}{rgb}{1.0,0.4,0.0}
\definecolor{violet}{rgb}{0.619608, 0.286275, 0.631373}
\definecolor{mybackgroundcolor}{rgb}{1.0,1.0,1.0}

%\definecolor{light}{rgb}{.5,0.5,0.0}
\definecolor{light}{rgb}{.3,0.3,0.3}

% sets backgroundcolor for whole document 
\pagecolor{mybackgroundcolor}
% sets text color
%\color{black}
% see below for an example how to change just a few words
% using \textcolor{color}{text}

\font \riesig=cmssbx10 scaled 13000
\font \gigant=cmssbx10 scaled 12000
\font \gross=cmssbx10 scaled 7000
\font \mittel=cmssbx10 scaled 5000
\font \courier=pcrb scaled 2000

\newcommand{\myFooter}{}
%\begin{picture}(0,0)(0,0)
%	\put(0,-185){\pol}
%\end{picture}}
\newcommand{\myNewSlide}{\newpage\myFooter} % \myBackground}

\usepackage{url}
\usepackage{hyperref}
\hypersetup{backref,  linkcolor=black, citecolor=black, colorlinks=true, hyperindex=true}
\renewcommand{\Pr}{\mathbb{P}}
\usepackage{pdfpages}
\begin{document}

\unitlength=1mm



\myNewSlide
\section*{Another toy example}
You tag 1000 territorial animals with transmitting tags.

Every month you survey the area.
Assume that you can detect every tag attached to a living organism.

You know (from other studies) that the probability of a tag falling off are:
0.10 in the first month, 0.15 in the second month, 0.2 in the third month, and
0.25 for every month after that.

Can you estimate the per-month probability of death?

\myNewSlide
You have studied bill width in a population of finches for many years. 

You have a standardization technique that converts the widths to standardized
widths that follow a Normal distribution with $\mu=0$ and $\sigma=1$. 
The standardization relies on your historical studies of the finches.

There is a drought, and you want to know if the mean width has changed.

\begin{center}
\begin{tabular}{c|c}
Indiv. & standardized bill width \\
\hline
1 & 0.01009121 \\
2 & 3.63415088 \\
3 &-1.40851589 \\
4 & 3.70573177 \\
5 &-0.94145782
\end{tabular}
\end{center}


\myNewSlide
\Large
Numerical optimization -- minimizing a function by evaluating it at many trial points.

Main points:
\begin{compactenum}
  \item optimizers can fail to find the global optimum:
  \begin{compactenum}
    \item multiple modes are a problem.
    \item result is often starting point dependent.
  \end{compactenum}
  \item limited precision in computers $\rightarrow$ rounding error, which complicates termination criteria.
\end{compactenum}

\myNewSlide
\Large
Numerical optimization -- practical recommendations.
\begin{compactenum}
  \item Try multiple starting points.
  \item Try multiple optimization algorithms.
  \item Don't skimp on optimization in your parametric bootstrapping (or at least make sure that the search for the global optimum is a good search).
  \item Reparameterization can help
  \item Using derivatives from finite differences can be surprisingly effective -- consider BFGS even if you can't calculate the gradient.
\end{compactenum}

\myNewSlide
\subsection*{Summary of LRT example}
\normalsize
\begin{itemize}
  \item A test based on the likelihood ratio test statistic is the most powerful hypothesis test.
  \item If we do not know the value of a parameter that occurs in the likelihood equation, we can estimate it.
  \item Even if we don't care about the parameter (e.g. $\sigma$ in our original question); its value {\em can} affect our hypothesis tests. 
  \item When the likelihood around the MLE looks ``normalish'' (not at a boundary and not a weird likelihood function), then the the $\chi_k^2$ distribution does a nice job of describing the null distribution of the LRT statistic for nested models.
\end{itemize}
\myNewSlide
\section*{Alternative forms of model selection}
The following methods do not assume that models are nested:
\subsection*{minimizing the Akaike Information Criterion}
  $$AIC(M|X) = 2k - 2 \ln L(\hat\theta | X, M)$$

\subsection*{Bayes Factors}
$B_{01}$ is the Bayes factor in favor of model 0 over model 1:
$$B_{01} = \frac{\Pr(X|M_0)}{\Pr(X|M_1)}$$
This is just a likelihood ratio, but it is not the likelihood evaluated at it maximum, rather it is:
\begin{equation}
  \Pr(X|M_0) = \int \Pr(X|\theta_0)\Pr(\theta_0) d\theta_0) \label{margLike}
\end{equation}
where $\theta_0$ is the set of parameters in model 0.

\myNewSlide
Bayes factors can be approximated using differences in:
  $$BIC(M|X) = 2k\ln(n) - 2 \ln L(\hat\theta | X, M)$$

Better approximations of the Bayes factor are available, but they are usually much more expensive.


\myNewSlide
\normalsize
You suspect that a population of big horn sheep are made up of two classes of males based on their fighting ability: Strong and Weak. The proportion of strong individuals is unknown.\\ {\bf Experiment:}
\begin{compactitem}
  \item You randomly select 10 pairs of males from a large population. 
  \item For each pair you randomly assign one of them the ID 0 and the other the ID 1.  
  \item You record the \# of winner from 2 contests.
\end{compactitem}
{\bf Model:}
\begin{compactitem}
  \item If two individuals within the same class fight, you expect either outcome to be equally likely.
  \item If a Strong is paired against a Weak then you expect that the probability that the stronger one wins with some probability, $w$.
  \item $w$ is assumed to be the same for every pairing of Strong {versus} Weak and the same for every fight within such a pairing.
\end{compactitem}

\myNewSlide
\begin{center}
\begin{tabular}{|c|c|c|}
\hline
& \multicolumn{2}{c|}{winner}\\
Pair \# & fight 1 & fight 2 \\
\hline
1 & 1 & 1  \\
\hline
2 & 1 & 0  \\
\hline
3 & 0 & 1  \\
\hline
4 & 1 & 1  \\
\hline
5 & 0 & 0  \\
\hline
6 & 0 & 1   \\
\hline
7 & 1 & 1  \\
\hline
8 & 0 & 0  \\
\hline
9 & 1 & 0  \\
\hline
10 & 1 & 1   \\
\hline
\end{tabular}
\end{center}

What can we say about $w$?


\myNewSlide
\begin{center}
\begin{tabular}{|c|c|c|}
\hline
& \multicolumn{2}{c|}{winner}\\
Pair \# & fight 1 & fight 2 \\
\hline
1 & 1 & 1  \\
\hline
2 & 1 & 0  \\
\hline
3 & 0 & 1  \\
\hline
4 & 1 & 1  \\
\hline
5 & 0 & 0  \\
\hline
6 & 0 & 1   \\
\hline
7 & 1 & 1  \\
\hline
8 & 0 & 0  \\
\hline
9 & 1 & 0  \\
\hline
10 & 1 & 1   \\
\hline
\end{tabular}
\begin{tabular}{ll}
\multicolumn{2}{c}{$X$}\\
\\
$x_1 = 1$ & $x_{11} = 1$  \\
$x_2 = 1$ & $x_{12} = 0$  \\
$x_3 = 0$ & $x_{13} = 1$  \\
$x_4 = 1$ & $x_{14} = 1$  \\
$x_5 = 0$ & $x_{15} = 0$  \\
$x_6 = 0$ & $x_{16} = 1$  \\
$x_7 = 1$ & $x_{17} = 1$  \\
$x_8 = 0$ & $x_{18} = 0$  \\
$x_9 = 1$ & $x_{19} = 0$  \\
$x_{10} = 1$ & $x_{20} = 1$  \\
\end{tabular}
\end{center}

$$L(w) = \prod_{i=1}^{20} \Pr(x_i|w)$$


\myNewSlide
\begin{center}
\begin{tabular}{|c|c|c|}
\hline
& \multicolumn{2}{c|}{winner}\\
Pair \# & fight 1 & fight 2 \\
\hline
1 & 1 & 1  \\
\hline
2 & 1 & 0  \\
\hline
3 & 0 & 1  \\
\hline
4 & 1 & 1  \\
\hline
5 & 0 & 0  \\
\hline
6 & 0 & 1   \\
\hline
7 & 1 & 1  \\
\hline
8 & 0 & 0  \\
\hline
9 & 1 & 0  \\
\hline
10 & 1 & 1   \\
\hline
\end{tabular}
\begin{tabular}{ll}
\multicolumn{2}{c}{$X$}\\
\\
$x_1 = 1$ & $x_{11} = 1$  \\
$x_2 = 1$ & $x_{12} = 0$  \\
$x_3 = 0$ & $x_{13} = 1$  \\
$x_4 = 1$ & $x_{14} = 1$  \\
$x_5 = 0$ & $x_{15} = 0$  \\
$x_6 = 0$ & $x_{16} = 1$  \\
$x_7 = 1$ & $x_{17} = 1$  \\
$x_8 = 0$ & $x_{18} = 0$  \\
$x_9 = 1$ & $x_{19} = 0$  \\
$x_{10} = 1$ & $x_{20} = 1$  \\
\end{tabular}
\end{center}

$$\Pr(x_{11} = 1 | x_{1} = 1, w) \neq \Pr(x_{11} = 1 \mid x_{1} = 0, w)$$

$$L(w) = \prod_{i=1}^{10} \Pr(x_i|w)\Pr(x_{10+i}|x_{i}, w)$$


\myNewSlide
\begin{center}
\begin{tabular}{|c|c|c|}
\hline
& \multicolumn{2}{c|}{winner}\\
Pair \# & fight 1 & fight 2 \\
\hline
1 & 1 & 1  \\
\hline
2 & 1 & 0  \\
\hline
3 & 0 & 1  \\
\hline
4 & 1 & 1  \\
\hline
5 & 0 & 0  \\
\hline
6 & 0 & 1   \\
\hline
7 & 1 & 1  \\
\hline
8 & 0 & 0  \\
\hline
9 & 1 & 0  \\
\hline
10 & 1 & 1   \\
\hline
\end{tabular}
\begin{tabular}{l}
$Y$\\
\\
$y_1 = (1, 1)$   \\
$y_2 = (1, 0)$  \\
$y_3 = (0, 1)$  \\
$y_4 = (1, 1)$  \\
$y_5 = (0, 0)$  \\
$y_6 = (0, 1)$  \\
$y_7 = (1, 1)$  \\
$y_8 = (0, 0)$  \\
$y_9 = (1, 0)$  \\
$y_{10} = (1, 1)$  \\
\end{tabular}
\end{center}


$$L(w) = \prod_{i=1}^{10} \Pr(y_i|w)$$


\myNewSlide
0 = same ram wins both bouts\\
1 = different rams win\\
\begin{center}
\begin{tabular}{|c|c|c|}
\hline
& \multicolumn{2}{c|}{winner}\\
Pair \# & fight 1 & fight 2 \\
\hline
1 & 1 & 1  \\
\hline
2 & 1 & 0  \\
\hline
3 & 0 & 1  \\
\hline
4 & 1 & 1  \\
\hline
5 & 0 & 0  \\
\hline
6 & 0 & 1   \\
\hline
7 & 1 & 1  \\
\hline
8 & 0 & 0  \\
\hline
9 & 1 & 0  \\
\hline
10 & 1 & 1   \\
\hline
\end{tabular}
\begin{tabular}{l}
$Z$\\
\\
$z_1 = 0$   \\
$z_2 = 1$  \\
$z_3 = 1$  \\
$z_4 = 0$  \\
$z_5 = 0$  \\
$z_6 = 1$  \\
$z_7 = 0$  \\
$z_8 = 0$  \\
$z_9 = 1$  \\
$z_{10} = 0$  \\
\end{tabular}
\end{center}


$$L(w) = \prod_{i=1}^{10} \Pr(z_i|w)$$


\myNewSlide
\begin{center}
\begin{tabular}{|c|c|c|}
\hline
& \multicolumn{2}{c|}{winner}\\
Pair \# & fight 1 & fight 2 \\
\hline
1 & 1 & 1  \\
\hline
2 & 1 & 0  \\
\hline
3 & 0 & 1  \\
\hline
4 & 1 & 1  \\
\hline
5 & 0 & 0  \\
\hline
6 & 0 & 1   \\
\hline
7 & 1 & 1  \\
\hline
8 & 0 & 0  \\
\hline
9 & 1 & 0  \\
\hline
10 & 1 & 1   \\
\hline
\end{tabular}
\begin{tabular}{l}
$Z$\\
\\
$z_1 = 0$   \\
$z_2 = 1$  \\
$z_3 = 1$  \\
$z_4 = 0$  \\
$z_5 = 0$  \\
$z_6 = 1$  \\
$z_7 = 0$  \\
$z_8 = 0$  \\
$z_9 = 1$  \\
$z_{10} = 0$  \\
\end{tabular}
\end{center}


$$L(w) = \prod_{i=1}^{10} \Pr(z_i|w)$$
$$ A = \sum_{i=1}^{10}z_i = 4$$
$$L(w) = \Pr(Z=0|w)^{(n-A)}\Pr(Z=1|w)^{A}$$

\end{document}
